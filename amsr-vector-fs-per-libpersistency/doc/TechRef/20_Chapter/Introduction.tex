%===========================================================================
% Introduction
%===========================================================================
\chapter{Introduction}
\index{Introduction}
\label{chap:Introduction}
The functional cluster Persistency offers two different mechanisms to access
persistent memory, the Key-Value-Storage and the File-Proxy.

The Key-Value-Storage offers access to one or multiple Key-Value-Databases for
every adaptive application.

A File-Proxy provides functions to read and write from files.

\begin{figure}[h!]
	\centering
	\includegraphics[
		width=0.75\textwidth
	]
	{{architecture}.jpg}
	\caption{Interfaces to other stack components}
	\label{fig:architecture}
\end{figure}

%===========================================================================
% Content

%===========================================================================
\section{Content}
\label{Content}

Persistency consists of these the following headers:
\begin{itemize}
	\item the Key-Value-Storage,
	\item which stores Key-Value-Storage-Type,
	\item the File-Proxy-Accessor-Factory,
	\item defined persistency exceptions
\end{itemize}

%===========================================================================
% Facts
%===========================================================================
\section{Facts}
The File-Proxy-Accessor-Factory creates the instances of Read-, Write or
Read-Write-Accessors to handle file reading or writing.
As a user of the Key-Value-Storage it is up to you to sync the saved
Key-Value-Pairs to the hard drive (to persist the Key-Value-Storage).