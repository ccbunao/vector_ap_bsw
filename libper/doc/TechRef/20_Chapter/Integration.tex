%===========================================================================
\chapter{Integration}
\label{chap:Integration}
\index{Integration}

%===========================================================================
\section{Build Dependencies}
\label{sec:BuildDependencies}
\index{Build Dependencies}

To build Persistency the following components are required:
\begin{itemize}
	\item amsr-vector-fs-libvac
\end{itemize}
Please also note figure \ref{fig:architecture} on page
\pageref{fig:architecture}.

%===========================================================================
\section{Runtime Dependencies}
\label{sec:RuntimeDependencies}

%===========================================================================
\section{Files and include structure}
\label{sec:FilesAndIncludeStructure}

The header files can be included like this:

ara/per/<header\_name>

The <header\_name>s are specified in
\href{https://www.autosar.org/fileadmin/user\_upload/standards/adaptive/17-10/AUTOSAR\_SWS_Persistency.pdf}{AUTOSAR\_SWS\_Persistency}.

%===========================================================================
\section{Cmake}
\label{sec:Cmake}
\index{cmake}
For detailed usage of cmake see documentation available at
\href{cmake.org}{https://cmake.org/documentation}.

\subsection{Build}
\label{sec:Cmake:Build}
\index{Build}
The cmake project provided with this component has options as specified in
\ref{tab:cmakeOptions}.

\begin{table}[h!]
	\pgfplotstabletypeset[row sep=newline]{
		cmake option	  & Description       & Defaultvalue
		-DENABLE\_DOXYGEN & Vector internal use. Set to OFF & OFF
		-DBUILD\_TESTS & Vector internal use. Set to OFF & OFF
		-DENABLE\_STATIC\_ANALYSIS & Vector internal use. Set to OFF & OFF
	}
	\caption{cmake options}
	\label{tab:cmakeOptions}
\end{table}

\subsection{Exported cmake packages}{}
\label{sec:Cmake:Build:Packages}
The amsr-vector-fs-em-persistency cmake project exports packages which can be
used to compile against the exported
library.

\subsection{Usage}
\label{sec:Cmake:Build:Usage}
\index{Usage}
To use Persistency an application needs to include the headers and instantiate
the provided objects to get access to
the functions.

To use the KeyValueStorage, create a KeyValueStorage object and create entries
of type KvsType to store them in the
KayValueStorage. It is important to call the function
KeyValueStorage::SyncToStorage to persist the data stored in the
database. The example shown in \ref{lst:kvs-example} demonstrates the usage of
the KeyValueStorage. The key-value-pairs
are stored in json format.

\begin{lstlisting}[language=C++, caption={Example usage of the KeyValueStorage}, captionpos=b, label={lst:kvs-example}]
#include <ara/per/keyvaluestorage.h>

/* create a KeyValueStorage instance */
KeyValueStorage kvstorage("path_to_keyvaluestorage_folder");

/* create some KvsTypes to store them in the database */
KvsType int_type(1);
KvsType bool_type(false);
KvsType string_type("string");

/* insert the KvsTypes in the database */
kvstorage.SetValue("anint", int_type);
kvstorage.SetValue("abool", bool_type);
kvstorage.SetValue("astring", string_type);

/* test if data was stored and print the value */
if(kvstorage.HasKey("anint")){
    std::cout << kvstorage.GetValue("anint").GetSInt() << std::endl;
}

/* sync to storage */
kvstorage.SyncToStorage();

\end{lstlisting}

To create an Read-, Write- or ReadWriteAccessor object the
FileProcyAccessorFactory provides three functions
(CreateReadAccess, CreateWriteAccess and CreateReadWriteAccess). Use the return
value of these functions to access
files. Use the global function CreateFileAccessorFactory to get the Factory.
The files must already exist as they won't
be created with instantiation of the file accessors. Example usage is shown in
\ref{lst:fpaf-example}.

\begin{lstlisting}[language=C++, caption={Example usage of the FileProxyAccessorFactory and Read-, Write- and ReadWriteAccessor}, captionpos=b, label={lst:fpaf-example}]
#include <ara/per/fileproxyaccessorfactory.h>


/* create a FileProxyAccessorFactory */
FileProxyFactoryUnqiuePtr factory = CreateFileAccessorFactory("path_to_files");

/*  Create a ReadAccessor for reading */
FileProxyAccessUniquePtr<ReadAccessor> reader = factory->CreateReadAccess("file_to_read", BasicOperations::OpenMode::in);

std::string result;
reader->getline(result, '\n');

/*  Create a WriteAccessor for writing */
FileProxyAccessUniquePtr<WriteAccessor> writer = factory->CreateWriteAccess("file_to_write", BasicOperations::OpenMode::out);

std::string write_this("to be written");
writer->write(write_this.data(), write_this.size());

/*  Create a ReadWriteAccessor for writing and reading */
FileProxyAccessUniquePtr<ReadWriteAccessor> rw = factory->CreateRWAccess("file_to_write", BasicOperations::OpenMode::in | BasicOperations::OpenMode::out);

std::string write_this("to be written\n");
rw->write(write_this.data(), write_this.size());

/*  set position in file to start */
rw.seek(0);

std::string result;
rw->getline(result, '\n');

\end{lstlisting}